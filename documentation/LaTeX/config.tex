%for algorithm pseudocode
%\usepackage{amsmath}
\usepackage[linesnumbered,ruled]{algorithm2e}
\usepackage{amsmath, amssymb, amsthm}
\usepackage{graphicx}
\usepackage[utf8]{inputenc}
\usepackage[T1]{fontenc}
\usepackage{authblk}
\usepackage{url}
\usepackage{caption}
\usepackage{subcaption}
\usepackage{tabularx}
\usepackage{geometry}
\usepackage{multirow}
\usepackage{booktabs}
\usepackage{graphicx}
\geometry{margin=1in}
\usepackage{array}
\usepackage{tabularx}
\usepackage[export]{adjustbox}
\usepackage{import}
\usepackage{float}
%\usepackage[caption=false]{subfig}

%%ORT
%metadata para la ORT
\usepackage[unicode, pdftex]{hyperref}
\hypersetup{ 
 pdfauthor={Federico de Leon, Mauricio Repetto},
 pdftitle={Data Augmentation usando imágenes
sintéticas},
 pdfsubject={Tesis de Maestría},
 pdfkeywords={}
}

%%%%%%%%%%%%%%%%%%% http://tex.stackexchange.com/a/103290
%cambio de formato de titulos
\usepackage{titlesec}

\titleformat{\chapter}[display]
{\normalfont\huge\bfseries}{}{20pt}{\Huge}

\titleformat{\section}
{\normalfont\fontsize{20}{0}\bfseries}{\thesection}{1em}{}

\titleformat{\subsection}
{\normalfont\fontsize{18}{0}\bfseries}{\thesubsection}{1em}{}

\titleformat{\subsubsection}
{\normalfont\fontsize{15}{0}\bfseries}{\thesubsubsection}{1em}{}

\titleformat{\paragraph}
{\normalfont\fontsize{13}{0}\bfseries}{}{1em}{}%no se muestra el numero, para agregarlo incluir \theparagraph

\titleformat{\subparagraph}
{\normalfont\fontsize{11}{0}\bfseries}{}{1em}{}%no se muestra el numero, para agregarlo incluir \thesubparagraph

%espacios despues de titulos
%defaults en http://ctan.dcc.uchile.cl/macros/latex/contrib/titlesec/titlesec.pdf dentro de Standard Classes
\titlespacing*{\chapter}        {0pt}{10pt}{40pt}
\titlespacing*{\section}        {0pt}{3.5ex plus 1ex minus .2ex} {2.3ex plus .2ex}
\titlespacing*{\subsection}     {0pt}{3.25ex plus 1ex minus .2ex}{1.5ex plus .2ex}
\titlespacing*{\subsubsection}  {0pt}{3.25ex plus 1ex minus .2ex}{1.5ex plus .2ex}
\titlespacing*{\paragraph}      {0pt}{3.25ex plus 1ex minus .2ex}{1.5ex plus .2ex}
\titlespacing*{\subparagraph}   {0pt}{3.25ex plus 1ex minus .2ex}{1.5ex plus .2ex}

%%%%%%%%%%%%%%%%%%%%%%%%%%%%%%%%%%%%%%%%%
%%posición de la numeración
\usepackage{fancyhdr}
\fancypagestyle{plain}{% Redefining plain page style
  \fancyhf{} %clear all header and footer fields
  \fancyfoot[RO]{\thepage}
}%
\fancyhf{} %clear all header and footer fields
\fancyfoot[RO]{\thepage}
\renewcommand{\headrulewidth}{0pt}
\renewcommand{\footrulewidth}{0pt}
\pagestyle{fancy}

\setlength{\parskip}{1em} % Add space between paragraphs


%%para los capítulos
\makeatletter
\def\@makechapterhead#1{%
  \vspace*{50\p@}%
  {\parindent \z@ \raggedright \normalfont
    \ifnum \c@secnumdepth >\m@ne
        \huge\bfseries \space \thechapter\space
    %    \par\nobreak
    %    \vskip 20\p@
    \fi
    \interlinepenalty\@M
    \Huge \bfseries #1\par\nobreak
    \vskip 40\p@
  }}
  \makeatother

%%Nombre del índice tabla de contenido
%%Si babel está activo lo pisa
\renewcommand{\contentsname}{Table of Contents}

%%para incluir imagenes
\usepackage{graphicx}

%%para la bibliografía
%%cambia el nombre
%%\renewcommand\bibname{Referencias Bibliográficas}
%%la agrega al indice
\usepackage[nottoc,numbib]{tocbibind}

%%%%%%%%%%%%%%%%%% impide que se separe un parrafo en dos paginas si es menor a 3 lineas
%http://tex.stackexchange.com/a/21985
\widowpenalties 4 10000 10000 5000 0
\raggedbottom

%%%%%%%%%%%%%%%%%% permite usar ifthenelse
%http://tex.stackexchange.com/a/58629
\usepackage{xifthen}

%%%%%%%%%%%%%%%%% referencias con nombre y numero
%http://tex.stackexchange.com/a/121871
\newcommand*{\fullref}[2][]{
\ifthenelse{\equal{#1}{}}
{\hyperref[{#2}]{\ref*{#2} \nameref*{#2}}}
{\hyperref[{#2}]{\ref*{#2} {#1}}}
}

%%%%%%%%%%%%%%%% proxima linea en \item y otros casos
\newcommand*{\nextLine}[0]{\mbox{}\\}