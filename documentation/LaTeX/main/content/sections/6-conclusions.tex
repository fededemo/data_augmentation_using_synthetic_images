\chapter{Conclusions}

Our research set out to investigate the use of advanced AI methods for generating synthetic data, with a particular focus on fields such as medicine, where a shortage of data can hinder the effectiveness of AI models. In partnership with Cogniflow, our goal was to enhance the performance of various image classification models by employing generative techniques to tackle issues related to insufficient training data and imbalanced classes.

Our findings revealed the complexity and diversity of challenges in applying generative models to different datasets. Synthetic data has immense promise for AI and analytics, but our study found that it is challenging to achieve satisfactory data augmentation without fully fine-tuning or retraining the models. Not doing so is essential for our objective of using them in production, as it allows automation for more generalized application and scalability. However, this approach presented difficulties in most analyzed datasets, where concepts are barely known by the techniques studied. The models' original weights, possibly trained on common datasets like ImageNet or others with similar concepts, may not readily adapt to the unique characteristics of specialized medical datasets.

One notable exception was the Chongqing dataset, where a small improvement was achieved. This dataset is characterized by its simplicity, with all the images bearing a strong similarity, in particular including the same chinese character in all the X-rays. As detailed earlier in our study, this uniformity is not common in the other datasets we worked with, and it likely contributed to the distinct results obtained with it.

The project's journey was not without challenges, reflecting the complexity of the field. These challenges included the intricate nature of generative models, the diversity of data across different datasets, the substantial computing resources required, the potential for bias in both the data and the models, and not insignificantly, the constant emergence of new findings, models, and techniques — almost on a weekly basis — throughout the course of our work. Despite these obstacle, the insights we have gathered help to highlight the complex relationship between synthetic data and its use in the real world applications, emphasizing the need for careful consideration, potential fine-tuning, and continued research in this area. 

In summary, this research underscores the significance of considering dataset characteristics and the limitations of generative techniques without proper adjustment. The potential to transform the development and training of ML models with synthetic data in specialized datasets and domains is yet to be proved, however this work gives additional basis for ongoing exploration and progress in this rapidly evolving field.
