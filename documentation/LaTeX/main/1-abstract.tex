\newpage
{\huge\bfseries \space Abstract en Español}
\bigskip
\bigskip
%%%%%%%%%%%%%%%% Cuerpo del Abstract %%%%%%%%%%%%%%%%%%%

En una era donde la escasez de datos de entrenamiento, especialmente en ciertos dominios, plantea desafíos para los modelos de \textit{Machine Learning} (\textit{ML}), los datos sintéticos ofrecen una solución convincente. Aunque este problema es particularmente visible en enfermedades raras, de ninguna manera es exclusivo de ellas. Los datos reales, como reflejo de la realidad, no abarcan todas las posibles condiciones o eventualidades. Al complementarlo con datos sintéticos, podemos tener en cuenta casos extremos y condiciones no vistas, permitiendo aplicaciones de \textit{ML} donde la escasez de datos podría haber dado lugar a modelos inutilizables debido a sesgos o a escenarios raros o sin precedentes.

Para este estudio, nuestro objetivo es centrarnos en el área de la \textit{Computer Vision} (\textit{CV}), y por lo tanto, nos enfocamos en el uso de imágenes y en problemas de clasificación relacionados con ellas. Proponemos la generación de datos sintéticos a través de técnicas modernas generativas de \textit{Inteligencia Artificial} (\textit{IA}) para imágenes, como los modelos de difusión (cuya notoriedad ha crecido significativamente recientemente), como una alternativa a las técnicas convencionales de \textit{Data Augmentation}.

Corroboramos que los modelos entrenados con una combinación de datos reales y sintéticos pueden superar a los entrenados sólo con datos reales. Ésta mejora, sin embargo, mostró variaciones significativas dependiendo del conjunto de datos y de la técnica generativa empleada. En un dataset particular, caracterizado por su simplicidad y uniformidad, el rendimiento demostró ser bueno. Por otro lado, en conjuntos de datos más variados, raros o especializados, los desafíos en la adaptación de los modelos generativos resaltaron la necesidad de un análisis cuidadoso y, posiblemente, de ajustes finos. Además, el conocimiento previo de los modelos generativos en los conceptos utilizados en el \textit{fine-tuning} resultó ser crucial para obtener imágenes sintéticas de calidad. Esto subraya la importancia de entrenar los modelos generativos en una amplia variedad de datos para que puedan reproducir los conceptos necesarios de manera efectiva. En resumen, los datos sintéticos sirven en algunos casos como una herramienta valiosa para mejorar la eficiencia de los modelos de \textit{ML} en tareas de \textit{CV}, particularmente con conjuntos de datos complejos debido a las características específicas dentro de sus clases o el desbalanceo de las mismas.


%%%%%%%%%%%%%%%%%%%%%%%%%%%%%%%%%%%%%%%%%%%%%%%%%%%%%%%%
\bigskip
\bigskip
\bigskip
{\huge\bfseries \space Abstract in English}
\bigskip
\bigskip
%%%%%%%%%%%%%%%% Cuerpo del Abstract %%%%%%%%%%%%%%%%%%%

In an era where training data scarcity, especially in certain domains, poses challenges for \textit{Machine Learning} (\textit{ML}) models, synthetic data offers a compelling solution. While this problem is particularly visible in rare diseases, it's by no means exclusive to them. Real data, as a reflection of reality, does not encompass every possible condition or eventuality. By supplementing it with synthetic data, we can account for edge cases and unseen conditions, allowing \textit{ML} applications where data scarcity might have otherwise led to unusable models due to bias or rare or unprecedented scenarios.

For this study, we aim to focus on the area of \textit{Computer Vision} (\textit{CV}), and therefore, we focus on the use of images and classification problems related to them. We propose the generation of synthetic data through modern generative \textit{Artificial Intelligence} (\textit{AI}) techniques for images, such as diffusion models (whose notoriety has grown significantly recently), as an alternative to conventional \textit{Data Augmentation} techniques. 

We corroborated that models trained with a combination of real and synthetic data can outperform those trained solely with real data. However, this improvement showed significant variations depending on the dataset and the generative technique employed. In one particular dataset, characterized by its simplicity and uniformity, the performance proved to be good. On the other hand, with more varied, rare, or specialized datasets, the challenges in adapting generative models highlighted the need for careful analysis and possibly fine-tuning. Furthermore, generative model's prior knowledge of the concepts used in fine-tuning proved to be crucial for obtaining quality synthetic images. This emphasizes the importance of training generative models on a wide variety of data to effectively reproduce the necessary concepts. In summary, synthetic data can often serve as a valuable tool for enhancing the efficiency of \textit{ML} models in \textit{CV} tasks, particularly with complex datasets due to the distinct characteristics within their classes or the presence of class imbalance.


%%%%%%%%%%%%%%%%%%%%%%%%%%%%%%%%%%%%%%%%%%%%%%%%%%%%%%%%

\newpage